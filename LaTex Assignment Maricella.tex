\documentclass[11pt, oneside]{article}   	% use "amsart" instead of "article" for AMSLaTeX format
\usepackage{geometry}                		% See geometry.pdf to learn the layout options. There are lots.
\geometry{letterpaper}                   	% ... or a4paper or a5paper or ... 
%\geometry{landscape}                		% Activate for rotated page geometry
%\usepackage[parfill]{parskip}    		% Activate to begin paragraphs with an empty line rather than an indent
\usepackage{graphicx}				% Use pdf, png, jpg, or eps with pdflatex; use eps in DVI mode
\usepackage{amssymb}
\usepackage{amsmath}
\usepackage{tikz}
\usepackage{pgfplots}
\usepackage{biblatex}                    % Include biblatex for bibliography
\addbibresource{references.bib}         % Add your .bib file for references

\title{Sample Paper Using LaTeX}
\author{Maricella Velita}
\date{Sept 27, 2024}


\begin{document}

\maketitle

\begin{abstract}
This is a sample template of an Academic Paper using LaTeX. In this paper, I will include sections, references, tables, figures, math expressions, and math equations. The purpose of this document is to demonstrate proper formatting.
\end{abstract}

\section{Introduction}
This introduction provides an overview of the academic paper. It outlines the structure of the paper, which includes various components. 

At the end of this introduction, a roadmap is provided. Section 2 covers the Methods, Section 3 presents the results, and Section 4 concludes the paper with a discussion.

\section{Methods}
In this section, the main methods used in this research will be presented including two mathematical expressions and two mathematical equations.

\subsection{Equations}

\begin{equation}
	\frac{d}{dx}  \left(  x^3  +  3x^2  +  2x \right) = 3x^2  +  6x  +  2
	\label{eq:differentiation}  
\end{equation}

Equation \ref{eq:differentiation} represents the differentiation of a cubic function.

Here is an example of an integral equation:
\begin{equation}
 	\int_0^1  x^2 \,  dx = \frac{1}{3}
	\label{eq:integral}
\end{equation}

Equation \ref{eq:integral} shows the evaluation of the integral of \(x^2\) from 0 to 1.

\subsection{Expressions}

In-line math expressions are used within a sentence. Here are some examples:
- The sum of two variables \( a + b \).
- A quadratic expression \( ax^2 + bx + c \).

\section{Results}
This section demonstrates how to present the results in a LaTeX document. In this section, a table and figure will be presented to show the results of the research.

\subsection{Table}
\begin{table}[h]
	\centering
	\begin{tabular}{|c|c|c|}
	\hline
	Statistic & Variable 1 & Variable 2 \\
	\hline
	Mean & 10.5 & 7.8 \\
	\hline
	Standard Deviation & 2.3 & 1.5 \\
	\hline
	\end{tabular}
	\caption{Summary Statistics for Variable 1 and Variable 2}
	\label{tab:summary_stats}
\end{table}

As shown in Table \ref{tab:summary_stats}, the mean and standard deviation for both variables are shown. 

\subsection{Figure}
\begin{figure}[h]
	\centering
	\begin{tikzpicture}
	\begin{axis}[
		boxplot/draw direction=y,
		ylabel={Values},
		xlabel={Category},
		xtick={1},
		xticklabels={Sample Data},
		title={Sample Boxplot}
	]
	\addplot+[
		boxplot prepared={
			lower whisker=5,
			lower quartile=10,
			median=15,
			upper quartile=20,
			upper whisker=25
		},
	]  coordinates  {};
	\end{axis}
	\end{tikzpicture}
	\caption{Boxplot for Sample Data}
	\label{fig:boxplot}
\end{figure}

As shown in Figure \ref{fig:boxplot}, this boxplot shows the distribution of the sample data.

\section{Conclusion}
This paper went over how to write an Academic Statistics Paper using LaTeX.

\appendix
\section{Appendix}
This is the appendix section of the paper.

\printbibliography

\end{document}